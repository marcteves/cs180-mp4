\documentclass{article}
\usepackage{graphicx}
	\setkeys{Gin}{width=1.0\textwidth}
\usepackage[a4paper]{geometry}
\usepackage{hyperref}
\usepackage{xcolor}
\usepackage{amsmath}
\definecolor{sexygray}{HTML}{1F1F1F}
\begin{document}
	\title{CS 180 MP 4: Artificial Neural Networks and Support Vector Machines}
	\author{Marc Teves\\ 2015-08007\\ THR}
	\maketitle

	Requirements:
	\begin{itemize}
		\item Python 3
		\item \texttt{numpy} python module
		\item \texttt{scikit-learn} python module
		\item \texttt{opencv} python module
		\item \texttt{bash} or your favorite shell
		\item \texttt{plotly} if you want to generate the graphs
	\end{itemize}

	Download \texttt{http://www.cl.cam.ac.uk/Research/DTG/attarchive/pub/data/att\_faces.zip}.

	Extract the \texttt{orl\_faces} folder to the project directory. Remove 
	\texttt{orl\_faces/README}. 

	Just run \texttt{main.py train\_list train\_tags test\_list test\_tags} or
	\texttt{start\_script.sh} to save on typing.

	\begin{enumerate}
		\item 
			The MLP classifier with half the number of the input nodes in the hidden layer was 94.3750\% accurate on the testing set.
		\item 
			Refer to Figure~1.
			In general, the MLP performs better with more nodes in the hidden layer.
			It achieves a maximum accuracy at around 8k nodes.
		\item 
			The MLP performed way worse (5\% accuracy), but it trained faster.
		\item
			The SVM trained really quickly, and had 95.6250\% accuracy on the test set.
		\item 
			Refer to Figure~2.
			The performance of the SVM did not change and it completed training at around the same time.
		\item
			Refer to Figure~3.
			The accuracy suffered rapidly as the gamma increased.
		\item
			The SVM trained much, much faster than the MLP and seemed to require less memory.
			The MLP was able to achieve a higher accuracy than the SVM, by finding the optimal number of nodes in the hidden layer.
	\end{enumerate}

	\begin{figure}[h]
		\centering
		\includegraphics{fig_1.png}
		\caption{Plot of accuracy vs. number of hidden layer nodes.}
	\end{figure}
	\begin{figure}[h]
		\centering
		\includegraphics{fig_2.png}
		\caption{Plot of accuracy vs. polynomial degree. Accuracy stayed the same.}
	\end{figure}
	\begin{figure}[h]
		\centering
		\includegraphics{fig_3.png}
		\caption{Plot of accuracy vs. gamma values. Accuracy decreases 
		exponentially faster with higher gamma values.}
	\end{figure}

\end{document}
